\documentclass[twosided,a4paper]{article}           %type
\usepackage[top=2cm,bottom=2cm,inner=1.5cm,outer=1.5cm]{geometry}       %geometry
\renewcommand{\familydefault}{\sfdefault}


%\usepackage[italian]{babel}                %font/language
\usepackage[T1]{fontenc}
\usepackage[utf8]{inputenc}

\usepackage{amsmath}                       %maths
\usepackage{bm}
\usepackage{amssymb}					   %for numbersets

\usepackage{graphicx}
\usepackage{epstopdf} 
\usepackage{float}
\usepackage{subfigure}

\usepackage{tabularx}

%\usepackage{hyperref}

%\usepackage{subfloat}
%full packaging for image in eps finally
\usepackage{textcomp}

\usepackage{color}
\usepackage[dvipsnames]{xcolor}

\usepackage{listings} 

\newcommand{\tr}{^{\tiny{\bm \top}}}

\newenvironment{sistema}%
{\left\lbrace\begin{array}{@{}l@{}}}%
	{\end{array}\right.}

\begin{document}
	
	\title{\textcolor{MidnightBlue}{Mechanical vibration} - \textcolor{Plum}{System identification and modal analysis of 3-DOF linear system}}
	\author{Giammarco Valenti}
	\maketitle
	
\section{Dynamical system}
\subsection{The system and the experimental setup}
The system consists of three different bodies on three carriers. The carriers are aligned and the bodies are constrained to slide along the common axis. Between the bodies two springs are located, and a third spring connects the frame and the last body. The first body is rigidly connected trough a rack-pinion gearing with a motor which is controlled in voltage with a PC interface. The position of each body is provided by an encoder. The zeroes of the positions are at the springs rest position. The scheme of the model is depicted in Figure \ref{fig:theplant1}.
The displacement of the masses in meters is available from the encoder with the following resolution:
\begin{equation}
	\Delta x = \dfrac{2 \pi r_e}{16000}.
	\label{eq:enc_res}
\end{equation}
\subsection{The dynamical model}
\subsubsection{Assumptions}
\label{sec:ass}
To define a model of the system, some assumptions were made:
\begin{description}
	\item[Rectilinear motion] All the bodies (masses), and the rack of the rack-pinion gearing are supposed to move and exert forces along the same axis, which is the motion axis. Consequently, all the quantities are meant to be projected on this axis.
	\item[Viscous friction] Only viscous frictions are present
	\item[Instantaneous electrical dynamics] The model of the electrical dynamics of the motor is only as a gain from voltage to force, expressed by the voltage-to-force factor, which is exploited in Section \ref{sec:step}.
	\item [Motor mechanics merged] Inertia and damping of the motor are merged respectively into $m_1$ and $c_1$ (refer to Figure \ref{fig:theplant1})
	\begin{equation}
	\begin{sistema}
	m_1 = m_{block} + \frac{J_{motor}\vert_{zz}}{r^2}\\
	c_2 = c_{block} + \frac{c_{motor}}{r^2}
	\end{sistema}
	\end{equation}
	where $r$ is the radius of the gear-rack coupling (gear wheel), $J_{motor}\vert_{zz}$ is the inertia of the motor , $c_{motor}$ the rotational damping and ''block'' quantities are the ones stricly related to the physical first mass.
	\item [Last viscosities merged] The term $c_3$ (refer to Figure \ref{fig:theplant1}) contains the viscous friction with the ground and the one due to the spring. In the model those two contributions cannot be quantified separately.
\end{description}
\subsubsection{The linear model}



The chosen model is a linear plant consisting of 3 lumped masses, 3 lumped springs between them (the last to the frame), and 3 dampers between each mass and the ground. The model is shown in Figure \ref{fig:theplant1}.
\begin{figure}[H]
	\centering
	\includegraphics[width=\linewidth]{img/theplant1}
	\caption[The linear plant]{The chosen plant, in red the unknown parameters}
	\label{fig:theplant1}
\end{figure} %TODO figure with displacements x
\subsection{equation of motion}
\begin{equation}
	\begin{sistema}
	m_1 \ddot{x}_1 = + k_1 \left (x_2 - x_1 \right )                                 + c_{12} \left ( \dot{x}_2 - \dot{x}_1 \right )                                                  - c_1\dot{x}_1 + g_v v\\
	m_2 \ddot{x}_2 = + k_1 \left (x_1 - x_2 \right ) + k_2 \left (x_3 - x_2 \right ) + c_{12} \left ( \dot{x}_1 - \dot{x}_2 \right ) +  c_{23} \left ( \dot{x}_3 - \dot{x}_2 \right ) - c_2\dot{x}_2\\
	m_3 \ddot{x}_3 =                                 + k_2 \left (x_2 - x_3 \right )                                                 +  c_{23} \left ( \dot{x}_2 - \dot{x}_3 \right ) - c_3\dot{x}_3 - k_3 x_3
	\end{sistema}
\end{equation}
In the classical matrix form:
\begin{subequations}
\begin{equation}
	\bm M \ddot{x} + \bm C \dot{x} + \bm K x = \bm b
	\label{eq:eqm:tot}
\end{equation}
where:\\
	\begin{tabularx}{\linewidth}{@{}XXX@{}}
	\begin{equation}
\bm K = \left[ \begin{array}{ccc}
+ k_1  & - k_1 & 0 \\ 
-k_1 & + k_1 + k_2 & -k_2 \\ 
0 & - k_2 & + k_3
\end{array}  \right]
\label{eq:eqm:K}
\end{equation} &
	\begin{equation}	
		\bm M = \left [\begin{array}{ccc}
		m_1 & 0 & 0 \\ 
		0 & m_2 & 0 \\ 
		0 & 0 & m_3
	\end{array} \right ]
	\label{eq:eqm:M}
\end{equation} \\
	\begin{equation}
\bm b = \left[ \begin{array}{ccc}
g_v \\ 0 \\  0
\end{array}  \right]
\label{eq:eqm:b}
\end{equation} &
	\begin{equation}
	\bm C = \left[\begin{array}{ccc}
	+ c_1 + c_{12} & -c_{12} &  0 \\ 
	-c_{12} & +c_2 + c_{12} + c_{23} & -c_{23} \\ 
	0 & -c_{23} & c_3 + c_{23} 
	\end{array} \right]
	\label{eq:eqm:C}
	\end{equation}
	\end{tabularx}
\label{eq:Mtot}
\end{subequations}
\subsection{state-space model}
	The linear model of the plant, expressed by the equation \ref{eq:Mtot}, is a SIMO model. A state-space form was chosen to represent this model. The matrices are the follwing:\\
\begin{subequations}
\begin{tabularx}{\linewidth}{@{}XXX@{}}
	\begin{equation}
		A = \left [ \begin{array}{cc}
		\bm Z_{3 \times 3} & \bm I_{3 \times 3} \\ 
		\bm{-M^{-1}K} & \bm{-M^{-1}C}
		\end{array} \right ]
		\label{eq:ss:A}
	\end{equation} &
	\begin{equation}
	\bm B =	\left [ \begin{array}{c}
		\bm Z_{3\times 1} \\ 
		\bm{-M^{-1}b}
		\end{array} \right ]
		\label{eq:ss:B}
	\end{equation} \\
		\begin{equation}
	\bm C =	\left [\begin{array}{cc}
	\bm I_{3\times 3} & \bm Z_{3\times 3}
	\end{array}   \right ]
	\label{eq:ss:C}
	\end{equation} &
	\begin{equation}
	\bm D =	\left [\begin{array}{c}
	\bm Z_{3 \times 1}
	\end{array}   \right ]
	\label{eq:ss:D}
	\end{equation} 
\end{tabularx}
\label{eq:ss}
\end{subequations}
where $\bm I$ is the identity matrix and $\bm Z$ is a matrix with all the entries equal to zero.
\subsection{experimental setup}
\subsubsection{data processing}
Few operations on data must be performed in order to use them. At first, the data on diplacements is provided in \textit{encoder counts}. They are converted in meters with the following convertion factor ($g_x$):
\begin{equation}
	g_x = \dfrac{\Delta x}{\Delta \texttt{counts}} = 2 \pi r_e \cdot \dfrac{\Delta \texttt{counts}}{16000\dfrac{\texttt{counts}}{\texttt{encoder revolution}}}
\end{equation}
	\subsection{parameters and data available}

\subsection{initial hypothesis and approximations}
\begin{itemize}
	\item neglected motor electrical dynamics (instantaneous transmission of torque)
	\item rectilinear motion (all perfect aligned)

	
\end{itemize}
\section{System identification}
	
\subsection{step response analysis}
\label{sec:step}
First of all, the step response analysis can be performed. In this analysis the ''static'' coefficients can be estimated, they are:
\begin{itemize}
	\item voltage to force $g_v$
	\item springs' stiffness $k_i$ with $i \in 1,2,3$
\end{itemize} 
The coefficient to be estimated is the \textcolor{Plum}{\textit{voltage-to-force}} coefficient %TODO SYMBOL 
\begin{equation}
	f = \textcolor{Plum}{\bm{(k_a\cdot k_t \cdot k_{mp})}}v = \textcolor{Plum}{g_v} v
\end{equation}.
In order to estimate the parameters the static gain vector $g_{dc}$ of the system has to be computed. The standard procedure is to apply the ''CAB'' formula from the state space formulation:
\begin{equation}
	\bm{g_{dc}} =	\bm{CA^{-1}B}
\end{equation}
which is the transfer function at $s = 0$. Since this computation implies the inverse of the $6 \times 6$ matrix $\bm A$, another computation is performed. Using the formulation in Equation \ref{eq:Mtot}, we can perform the following limits:
\begin{equation}
	\begin{sistema}
	\lim_{t \to + \infty}{\dot{x}}  = 0  \\
	\lim_{t \to + \infty}{\ddot{x}} = 0
	\end{sistema}
	\label{eq:dot_limits}
\end{equation}
The substitution \ref{eq:dot_limits} in \ref{eq:eqm:tot} yields:
\begin{equation}
\bm K x = \bm b
\label{eq:dot0}
\end{equation} The static gain is then:
\begin{equation}
	\bm{g_{dc}} = \bm{K^{-1}}\bm{b}
\end{equation}
Note that it is equivalent to apply the Laplace tranform to the equation \ref{eq:eqm:tot} and apply the final value theorem to it.
\begin{equation}
	\bm{g_{dc}} =
	\left [ \begin{array}{ccc}
	g_v\dfrac{k_1k_2 + k_3k_2 + k_3k_1}{k_1k_2k_3}  & g_v\dfrac{k_2+k_3}{k_2k_3} & g_v\dfrac{1}{k_3}
	\end{array} \right ]\tr
	\label{eq:g_dc}
\end{equation}
Note that in equation \ref{eq:g_dc} is evident from the expression the parallel between the stiffnesses (at steady state inertia and damping are invisible).
The steady state value of the three output are available. Some other computations has to be made to make equation \ref{eq:g_dc} suitable for the check on the stiffnesses ratios and the new estimation of $g_v$. The equation for the steady state values is:
\begin{equation}
	\bm{C}\bm{x}_\infty = \bm{g_{dc}}u_\infty  
	\label{eq:stst1}
\end{equation}
Where the $\infty$ denotes the steady state value of the quantity. The Equation \ref{eq:stst1} is now expressed in terms of stiffnesses ratio: $k_3$ is fixed to the nominal value and two ratio are defined:\begin{itemize}
	\item fix $k_3$ on nominal value
	\item multiplied both sides times $k_3$
	\item replace $R_{32} = \dfrac{k_3}{k_2}$
	\item replace $R_{31} = \dfrac{k_3}{k_1}$
\end{itemize}
This operations are made in order to make the system easy to solve, and uncoupling the nonlinear factor $g_v$. This allow to make to solve the system in a cascade fashion from $g_v$ to $R_{13}$. The equation \ref{eq:stst1} is now expressed in the following form:
\begin{equation}
	\label{eq:stst2} 
	k_3 \left [\begin{array}{c}
	x_1(\infty) \\ 
	x_2(\infty) \\ 
	x_3(\infty) \end{array}  \right ] 
	=
	u(\infty)\left [\begin{array}{c}
	g_v + g_v R_{13} + g_v R_{23} \\ 
	g_v R_{23} \\ 
	g_v \end{array}  \right ] 
\end{equation}
Now the next step is to use the measured value of steady state response (input and output), take $g_v$ as new estimated value and verify that $R_{12}$ and $R_{32}$ matches the nominal values. As set before $u(\infty) = 0.5$.
Results on the ratios are shown in Table \ref{tab:ratios}
%TODO< write the mambo jambo
\begin{table}[H]
	\centering
	\begin{tabular}{|c|c|c|c|}
		\hline 
		 & from data & nominal & error \% \\ 
		\hline 
		$k_3/k_2$ & \input{result/R32_exp.txt}& \input{result/R32_nom.txt} & \input{result/R32_per.txt}\%  \\ 
		\hline 
		$k_3/k_1$ & \input{result/R31_exp.txt}& \input{result/R31_nom.txt} & \input{result/R31_per.txt}\% \\
		\hline \hline
		      & from data              & initial & error 
		      \%\\
		\hline
		$g_v$ & \input{result/g_v.txt} & \input{result/gain_v.txt} & \input{result/g_v_per.txt}\% \\\hline
	\end{tabular} 
	\label{tab:ratios}
	\caption{Stiffnesses ratios and voltage-to-force coefficients results}
\end{table}
The $voltage-to-force$ estimation is shown in table \ref{tab:ratios}.
\subsection{Parameters estimation}
To estimate the parameters the impulse reponse is used. The \textit{voltage-to-force} coefficient $g_v$ is one of the parameters to be estimate in order to have the possibility to cross-check the result with the step response result.
\subsubsection{Free damping case}
The first estimation is performed using the model in Equations \ref{eq:ss}.
%TODO explain and add a block cheme with the steps!
\section{modal analysis}
\subsection{Eigenvalue problem}
\label{sec:eigenvalueproblem}
\subsection{Undamped case}
\subsubsection{Raileight quotient}
	The Rayleight quotient is 
	\begin{equation}
		R_q(\bm x) = \dfrac{\bm x\tr \bm K \bm x}{\bm x\tr \bm M \bm x} 
	\end{equation}
	where x is the vector of the positions, $R_q$ as a function of the ladder, presents stationary points in the neigborhood of the modal shapes, and the value corresponds to the respective eigenfrequency. The first stationary point it's easy to find, but the other two are not. In order to find the other stationary points, the property of orthogonality of the modal shapes vector is used. The basic idea is to reduce a degree of freedom every frequency using this property. The procedure is the following:
	\begin{itemize}
		\item Provide an initial guess, the convergence to the first minima is quite robust to the first guess. Vector $[\ 1 \ 1 \ 1 \ ]\tr$ is used.
		\item Use Matlab function \texttt{fminunc} to find the minima and the value of $x$, varying only two parameters of the vector ($x = [\ 1 \ \alpha \  \beta\ ]\tr$) 
		\item Check if the minimum is the lowest frequency with a contour plot or performing the next steps. Define the first modal shape vector $u_{1R}$.  
		\item Compute the null space of the first modal shape vector $Ker(u_{1R}\tr) $
		\item find the minimum of
		\begin{equation}
			R_{q2} = \dfrac{[ \ 1 \ \alpha \ ] B_k\tr \bm K B_k [ \ 1 \ \alpha \ ]\tr}{[ \ 1 \ \alpha \ ] B_k\tr \bm M B_k [ \ 1 \ \alpha \ ]\tr}
		\label{eq:rq2}
		\end{equation}
		where $img(B_k) = Ker(u_{1R}\tr)$ is a base of the Kernel of the first modal shape vector, the value of the function is another resonance frequency.
		\item The last modal shape vector and the last resonance frequency can be computed respectively as a base of the Kernel of the two other modal shape vectors and evaluating $R_q$ at the last modal shape vector.
		\begin{equation}
			u_{3R} \in Ker(\left [
			\begin{array}{cc}
			u_{1R} &  u_{2R}
			\end{array} \right ]\tr)
		\end{equation}
	\end{itemize}
The results are shown in Table \ref{tab:modalResults} and the contour plot of the Raileight coefficient is shown in Figure \ref{fig:Rcontour}
\subsection{Proportional damping case}
The mode shapes of the system remains the same but the frequencies will be different and complex (BIBLIO RAO PAG 914).

given the modal shape vectors matrix $U$ obtained from the eigenvalue problem \ref{sec:eigenvalueproblem} with the masses from the proportional damping estimation, the modal coordinates w.r.t the previous coordinates are the following:

\begin{equation}
	\tilde{x} = U\tr x 
\end{equation}
 Which is basically a projection of the coordinate on the modal shapes vector (scalar products).
 The matrices $M$, $K$ and $C$ becomes... %TODO input files
%\input{result/M_s_matrix_tex.txt}



%TODO tabella al 
\end{document}